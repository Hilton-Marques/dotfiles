Vim�UnDo��P`LJ7�@��2
!A��6@�XtVRM����Q$$f|�_�����f|��$$5��