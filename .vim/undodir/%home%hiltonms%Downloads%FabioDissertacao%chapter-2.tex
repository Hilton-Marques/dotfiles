Vim�UnDo��c���)8?�YLfݷ�Ǣ�*(��
fOd�_�U����fOd��
% -*- coding: utf-8; -*-$\chapter{Referência Bibliográfica}\label{chapter-referências}\section{Introdução}Um sistema de aeronave não tripulada (SANT) compreende um conjunto de subsistemas, incluindo o veículo aéreo não tripulado (VANT) propriamente dito, a carga paga, a estação de controle em terra (ECT), e outros conforme pode ser verificado na Fig. \ref{fig:diag_sant1}.�A diferença fundamental de um SANT para um sistema de aeronave tripulado reside nas interfaces e aviônica, que dão lugar a um conjunto de aviônica e subsistema de controle que farão a função da tripulação.Também não se confunde um VANT com modelos de aeronaves (ou aeromodelos), uma vez que estes últimos normalmente não possuem aviônica ou recursos de apoio ao vôo embarcados, resumindo-se a modelos em escala das suas versões originais em tamanho real.XAssim, o VANT do tipo multirrotor omnidirecional será o escopo dos tópicos que seguem.2\begin{figure}[!ht]%[!htb] = here or top or bottom    \centeringO    \includegraphics[scale=0.5]{images/cap-2_RevisaoBib/diagramageralsant1.png}6    \caption{Diagrama Geral de um SANT - Fonte: Autor}    \label{fig:diag_sant1}\end{figure}�%http://latexbr.blogspot.com/2011/07/inserindo-figuras-no-latex.html#:~:text=Inserindo%20as%20figuras&text=%5B!,ficar%C3%A1%20exatamente%20onde%20voc%C3%AA%20inseriu.\section{Estado da Técnica} �Uma vez que os Veículos Aéreos Não Tripulados (VANTs) do tipo multimotor são derivados de aeronaves de asa rotativa tripulada, sua configuração típica remete a sua congênere tripulada.�Assim, a geometria das aeronaves multimotores de configuração ordinária possuem arranjo estrutural de seus propulsores disposto em um plano em que a direção de empuxo é paralelo e com fluxo direcionado verticalmente.Essa plataforma é intrinsecamente instável, em alguns casos subatuada ou sobreatuada (à depender da quantidade de atuadores instalados), com a dinâmica fortemente não-linear, acoplada, multivariável, sujeita à incertezas, além de possuir alguns efeitos dinâmicos não previstos.�Contudo, essa solução já é um conceito de sistema com considerável difusão e com certa expectativa como pode ser visto na literatura do setor nas últimas duas décadas.�Os Veículos multirrotores são amplamente utilizados em aplicações robóticas por seu baixo custo, manobrabilidade e velocidade. No entanto, a incapacidade dos multirotores tradicionais de gerar empuxo e torque independentemente um do outro, e em qualquer direção, tem duas consequências principais. Primeiro, o conjunto de manobras viáveis para esses Veículos Aéreos Não Tripulados (VANTs) é severamente limitado devido ao acoplamento de posição e atitude, como mostrado por Bolandi \cite{BOLANDI2012}, Lyu \cite{LIUYU2013}, Fernandes \cite{FERNANDES2017}, Enram \cite{EMRAN2018165} e Beitel \textit{et al.} \cite{BEITELSCHMIDTetal2019}. Em segundo lugar, os VANTs são incapazes de resistir a distúrbios arbitrários de força e torque sem atrasos substanciais. Isto se deve ao tempo necessário para reorientar a direção do empuxo fixo no chassi do VANT, o que degrada o desempenho dos VANT em cenários que exigem voo de alta precisão e em cenários onde são encontradas grandes perturbações externas. Por exemplo, nas aplicações de física aérea de interação, ou em atividades como agarrar e manipular objetos ou interação humana.O desenvolvimento de um VANT multirrotor ou híbrido (que combina duas ou mais formas de sustentação) com capacidades omnidirecionais se deu à partir das adaptações das soluções já existentes e/ou conhecidas associadas ao emprego de diversas metodologias.�Considerando os trabalhos da última década, os desenvolvimentos se concentram ora na dotação de capacidade estrutural, ora na lógica/aviônica necessária  para o vôo com capacidades omnidirecionais.7No primeiro caso, as soluções se concentram ou na alteração estrutural para que seja possível a movimentação do rotor e assim atingir a postura desejada, ou adotar-se-á uma configuração ótima dos rotores para que exista uma capacidade máxima de potencial de atingir a totalidade de posturas possíveis dentro do volume de trabalho. Já no segundo caso técnicas e estratégias de controle são implementadas e melhoradas para que seja possível dotar uma plataforma comum ou dita ordinária com capacidade de alcançar o pleno potencial de movimentação.�Um exemplo de demanda por aumento de capacidades é um trabalho da década passada apresentado por Jung  \textit{et al.} \cite{JUNGetalE2014}, que apresenta um quadrirrotor adaptado para que este possa operar tanto em vôo, com os rotores na posição de topo, quanto em superfície, de forma semelhante a um robô móvel terrestre, porém com os rotores em posição de fluxo horizontal.j% \begin{figure}[H] %htbp (h) aqui, (t) topo, (b) base, ou na (p) página, ! - forçar, (H) - aqui "brabo"%     \centering#%         \subfigure[Modo de vôo. #%         \label{subfig:flymobileV}%         ]{T%         \includegraphics[scale=0.25]{images/cap-2_RevisaoBib/jungetal2014_fly.png}%         } % end subfigure'%         \subfigure[Modo de terrestre.#%         \label{subfig:flymobileT}%         ]{V%         \includegraphics[scale=0.25]{images/cap-2_RevisaoBib/jungetal2014_drive.png}%         } % end subfigure8%         %\quad % dá um espaço entre as duas figuras.M%     \caption{\textit{Flymobile} - modos de operação \cite{JUNGetalE2014}} %     \label{fig:flymobileModes}% \end{figure}% PARA COMENTAR GERAL "CTRL+/"% \begin{figure}[H]%     \centering)%     \begin{subfigure}[b]{0.4\textwidth}%         \centeringT%         \includegraphics[scale=0.25]{images/cap-2_RevisaoBib/jungetal2014_fly.png}!%         \caption{Modo de vôo.}#%         \label{subfig:flymobileV}%     \end{subfigure}%     \hfill)%     \begin{subfigure}[b]{0.4\textwidth}%         \centeringV%         \includegraphics[scale=0.25]{images/cap-2_RevisaoBib/jungetal2014_drive.png}&%         \caption{Modo de terrestre.}#%         \label{subfig:flymobileT}%     \end{subfigure}%     \caption{C% 			\textit{Flymobile} - modos de operação \cite{JUNGetalE2014}} %     \label{fig:flymobileModes}% \end{figure}\subimages{>\textit{Flymobile} - modos de operação \cite{JUNGetalE2014}.}{55}{T \subimage[ \cite{Modo de vôo} ]{.47}{images/cap-2_RevisaoBib/jungetal2014_fly.png}Y \subimage[\cite{Modo de terrestre.} ]{.50}{images/cap-2_RevisaoBib/jungetal2014_fly.png}}�Para as soluções de rotores móveis ou inclináveis, os rotores são direcionados conforme um algoritmo de alocação de controle, para assim alcançar a atitude desejada. Diversos trabalhos explorando essas propostas com esse tipo de estratégia foram sugeridas ao longo dessa última década, como pode ser verificado nos trabalhos de Kim \cite{MsThesisKIM2017}, de Kamel \textit{et al.} \cite{KAMELetal2018}, de Gilitschenski \textit{et al.} \cite{GILITSCHENSKIZetal2018}, de Piljek \textit{et al.} \cite{PILJEKetal2018}, de Jansheck \textit{et al.} \cite{JANSCHEKetal2018}, de Yang \textit{et al.} \cite{YANGetal2019}, de Kovac \textit{et al.} \cite{KOVACetal2020}, de Zhang \textit{et al.} \cite{ZHANGetal2020}, de Bodie \textit{et al.} \cite{BODIEetal2020}, de Lu \textit{et al.} \cite{LUandYANGetal2020}, de Chen e Jia \cite{CHENJIA2020} e de Gao \textit{et al.} \cite{GAOetal2020}. Em todos esses trabalhos o conceito geral de rotor direcionável é explorado, utilizando plataformas sub ou sobre-atuadas. sNo trabalho de Kim \cite{MsThesisKIM2017}, o conceito de rotor direcionável é abordado de forma semelhante ao trabalho de Jung \textit{et al.} \cite{JUNGetalE2014}e a diferença está na capacidade que a sua proposta possui, de assumir a postura a depender da situação. Entretanto, essa é uma proposta conceitual, sem construção de protótipo ou prova de conceito.h\begin{figure}[H] %htbp (h) aqui, (t) topo, (b) base, ou na (p) página, ! - forçar, (H) - aqui "brabo"    \centering6        \subfigure[Modo de vôo pairado "horizontal".         \label{subfig:kim1}]	        {F        \includegraphics[scale=0.45]{images/cap-2_RevisaoBib/kim3.png}        } % end subfigure5        \subfigure[Modos de vôo pairado "vertical".         \label{subfig:kim2}]	        {F        \includegraphics[scale=0.45]{images/cap-2_RevisaoBib/kim4.png}        } % end subfigure5        \quad % dá um espaço entre as duas figuras.:    \caption{Modos de vôo pairado \cite{MsThesisKIM2017}}    \label{fig:kimconcept}\end{figure}% \begin{figure}[H]%     \centering*%     \begin{subfigure}[b]{0.45\textwidth}%         \centeringH%         \includegraphics[scale=0.45]{images/cap-2_RevisaoBib/kim3.png}6%         \caption{Modo de vôo pairado "horizontal".}%         \label{subfig:kim1}%     \end{subfigure}%     \hfill*%     \begin{subfigure}[b]{0.45\textwidth}%         \centeringH%         \includegraphics[scale=0.45]{images/cap-2_RevisaoBib/kim4.png}5%         \caption{Modos de vôo pairado "vertical".}%         \label{subfig:kim2}%     \end{subfigure}<%     \caption{Modos de vôo pairado \cite{MsThesisKIM2017}}%     \label{fig:kimconcept}% \end{figure}Já o trabalho de Kamel \textit{et al.} \cite{KAMELetal2018} e de Gilitschenski \textit{et al.} \cite{GILITSCHENSKIZetal2018} explora o conceito de rotor direcionável à partir de uma plataforma típica com seis rotores denominada Voliro e estuda o problema de Alocação de Controle. h\begin{figure}[H] %htbp (h) aqui, (t) topo, (b) base, ou na (p) página, ! - forçar, (H) - aqui "brabo"    \centering&        \subfigure[Voliro - conceito.          \label{subfig:voliro1}]{K        \includegraphics[scale=0.55]{images/cap-2_RevisaoBib/Elkhatib2.png}        } % end subfigure;        \subfigure[Voliro em vôo. \label{subfig:voliro2}]{K        \includegraphics[scale=0.85]{images/cap-2_RevisaoBib/Elkhatib1.png}        } % end subfigure5        \quad % dá um espaço entre as duas figuras.I    \caption{Voliro - Detalhes conceituais - Fonte: \cite{KAMELetal2018}}    \label{fig:voliroconcept}\end{figure}2%\begin{figure}[!h]%[!htb] = here or top or bottom%    \centeringH%    \includegraphics[scale=0.5]{images/cap-2_RevisaoBib//Elkhatib4.png}F%    \caption{Flymobile em modo de vôo - Fonte: \cite{JUNGetalE2014}}!%    \label{fig:jungetal2014_fly}
%\end{figure}% \begin{figure}[H]%     \centering*%     \begin{subfigure}[b]{0.45\textwidth}%         \centeringM%         \includegraphics[scale=0.55]{images/cap-2_RevisaoBib/Elkhatib2.png}&%         \caption{Voliro - conceito.} %         \label{subfig:voliro1}%     \end{subfigure}%     \hfill*%     \begin{subfigure}[b]{0.45\textwidth}%         \centeringM%         \includegraphics[scale=0.85]{images/cap-2_RevisaoBib/Elkhatib1.png}#%         \caption{Voliro em vôo.} %         \label{subfig:voliro2}%     \end{subfigure}J%     \caption{Voliro - Detalhes conceituais - Fonte: \cite{ELKHATIB2017}}%     \label{fig:voliroconcept}% \end{figure}!Os trabalhos de Gilitschenski \textit{et al.} \cite{GILITSCHENSKIZetal2018}, de Piljek \textit{et al.} \cite{PILJEKetal2018} e de Jansheck \textit{et al.} \cite{JANSCHEKetal2018} abordam o problema de forma semelhante a Kamel \textit{et al.} \cite{KAMELetal2018}. O primeiro se concentra no próprio Voliro e desenvolve o estudo explorando soluções para o problema de Alocação de Controle e implementa simulações e testes. O segundo e o terceiro trabalhos exploram técnicas e estratégias de controle em plataformas semelhantes ao Voliro.h\begin{figure}[H] %htbp (h) aqui, (t) topo, (b) base, ou na (p) página, ! - forçar, (H) - aqui "brabo"    \centering(        \subfigure[Detalhes do Sistema.          \label{fig:flymobileV}]{K        \includegraphics[scale=0.75]{images/cap-2_RevisaoBib/Elkhatib3.png}        } % end subfigure        \subfigure[Modelagem.          \label{fig:flymobileT}]{K        \includegraphics[scale=0.45]{images/cap-2_RevisaoBib/Elkhatib4.png}        } % end subfigure6        %\quad % dá um espaço entre as duas figuras.Z    \caption{Voliro - Sistema de direcionamento dos motores - Fonte: \cite{KAMELetal2018}}    \label{fig:volirooperation}\end{figure}% \begin{figure}[H]%     \centering*%     \begin{subfigure}[b]{0.45\textwidth}%         \centeringM%         \includegraphics[scale=0.75]{images/cap-2_RevisaoBib/Elkhatib3.png}(%         \caption{Detalhes do Sistema.} %         \label{fig:flymobileV}%     \end{subfigure}%     \hfill*%     \begin{subfigure}[b]{0.45\textwidth}%         \centeringN%         \includegraphics[scale=0.45]{images/cap-2_RevisaoBib//Elkhatib4.png}%         \caption{Modelagem.} %         \label{fig:flymobileT}%     \end{subfigure}[%     \caption{Voliro - Sistema de direcionamento dos motores - Fonte: \cite{ELKHATIB2017}}!%     \label{fig:volirooperation}% \end{figure}j\begin{figure}[!ht] %htbp (h) aqui, (t) topo, (b) base, ou na (p) página, ! - forçar, (H) - aqui "brabo"    \centering(        \subfigure[Protótipo em teste. "        \label{fig:volirointest}]{K        \includegraphics[scale=0.45]{images/cap-2_RevisaoBib/Elkhatib5.png}        } % end subfigure&        \subfigure[Manobra possível. #        \label{fig:volirointest2}]{J        \includegraphics[scale=0.5]{images/cap-2_RevisaoBib/Elkhatib6.png}        } % end subfigure5        \quad % dá um espaço entre as duas figuras.=    \caption{Voliro - Manobras - Fonte: \cite{KAMELetal2018}}    \label{fig:voliromaneuver}\end{figure}% \begin{figure}[H]%     \centering*%     \begin{subfigure}[b]{0.45\textwidth}%         \centeringM%         \includegraphics[scale=0.45]{images/cap-2_RevisaoBib/Elkhatib5.png}(%         \caption{Protótipo em teste.}"%         \label{fig:volirointest}%     \end{subfigure}%     \hfill*%     \begin{subfigure}[b]{0.45\textwidth}%         \centeringM%         \includegraphics[scale=0.5]{images/cap-2_RevisaoBib//Elkhatib6.png}&%         \caption{Manobra possível.}#%         \label{fig:volirointest2}%     \end{subfigure}>%     \caption{Voliro - Manobras - Fonte: \cite{ELKHATIB2017}} %     \label{fig:voliromaneuver}% \end{figure} % e Elkhatib \cite{ELKHATIB2017}kApesar de propor uma plataforma "multidirecional", o trabalho de Kovac \textit{et al.} \cite{KOVACetal2020}�apresenta um conceito semelhante a Kim \cite{MsThesisKIM2017}, em que a sua solução muda a sua postura e desempenha as suas missões se utilizando da configuração dos rotores em relação ao eixo vertical do chassi.2\begin{figure}[!ht]%[!htb] = here or top or bottom    \centeringD    \includegraphics[scale=0.95]{images/cap-2_RevisaoBib/Kovac1.png}N    \caption{Conceito proposto por Kovac \textit{et al.} \cite{KOVACetal2020}}    \label{fig:kovac}\end{figure}xZhang \textit{et al.} \cite{ZHANGetal2020} estuda uma arquitetura semelhante aos trabalhos de Kamel \cite{KAMELetal2018}, de Gilitschenski \textit{et al.} \cite{GILITSCHENSKIZetal2018}, de Piljek \textit{et al.} \cite{PILJEKetal2018} e de Jansheck \textit{et al.} \cite{JANSCHEKetal2018}, e estuda os efeitos e a robustez do sistema quando esse opera com distúrbios de vento.j\begin{figure}[!ht] %htbp (h) aqui, (t) topo, (b) base, ou na (p) página, ! - forçar, (H) - aqui "brabo"    \centering(        \subfigure[Referenciais locais.         \label{fig:zhang1}]{G        \includegraphics[scale=0.5]{images/cap-2_RevisaoBib/Zhang2.png}        } % end subfigure'        \subfigure[Referencial global.         \label{fig:zhang2}]{G        \includegraphics[scale=0.5]{images/cap-2_RevisaoBib/Zhang1.png}        } % end subfigure5        \quad % dá um espaço entre as duas figuras._    \caption{Referenciais dos motores e visão geral da proposta de Zhang \cite{ZHANGetal2020}}    \label{fig:zhang}\end{figure}% \begin{figure}[!h]%     \centering)%     \begin{subfigure}[b]{0.5\textwidth}%         \centeringI%         \includegraphics[scale=0.5]{images/cap-2_RevisaoBib/Zhang2.png}(%         \caption{Referenciais locais.}%         \label{fig:zhang1}%     \end{subfigure}%     \hfill)%     \begin{subfigure}[b]{0.5\textwidth}%         \centeringI%         \includegraphics[scale=0.5]{images/cap-2_RevisaoBib/Zhang1.png}'%         \caption{Referencial global.}%         \label{fig:zhang2}%     \end{subfigure}a%     \caption{Referenciais dos motores e visão geral da proposta de Zhang \cite{ZHANGetal2020}}%     \label{fig:zhang}% \end{figure}2\begin{figure}[!ht]%[!htb] = here or top or bottom    \centeringC    \includegraphics[scale=0.5]{images/cap-2_RevisaoBib/Zhang3.png}S    \caption{Análise aerodinâmica no sistema moto-propulsor \cite{ZHANGetal2020}}    \label{fig:zhang3}\end{figure}"O trabalho de Bodie \textit{et al.} \cite{BODIEetal2020} propõe uma nova solução que emprega o dobro de motores que os utilizados em propostas semelhantes, motivado por estudos para o arranjo ótimo, com o propósito de ser robusto à singularidades, omnidirecional e de vôo eficiente. h\begin{figure}[H] %htbp (h) aqui, (t) topo, (b) base, ou na (p) página, ! - forçar, (H) - aqui "brabo"    \centering,        \subfigure[Conceito e referenciais.         \label{fig:bodie1}]{H        \includegraphics[scale=0.35]{images/cap-2_RevisaoBib/Bodie2.png}        } % end subfigure        \subfigure[Protótipo.         \label{fig:bodie2}]{I        \includegraphics[scale=0.325]{images/cap-2_RevisaoBib/Bodie1.png}        } % end subfigure5        \quad % dá um espaço entre as duas figuras.G    \caption{Modelo de multirrotor proposto Bodie \cite{BODIEetal2020}}    \label{fig:bodieconcept}\end{figure}% \begin{figure}[H]%     \centering*%     \begin{subfigure}[b]{0.45\textwidth}%         \centeringJ%         \includegraphics[scale=0.35]{images/cap-2_RevisaoBib/Bodie2.png},%         \caption{Conceito e referenciais.}%         \label{fig:bodie1}%     \end{subfigure}%     \hfill*%     \begin{subfigure}[b]{0.45\textwidth}%         \centeringK%         \includegraphics[scale=0.325]{images/cap-2_RevisaoBib/Bodie1.png}%         \caption{Protótipo.}%         \label{fig:bodie2}%     \end{subfigure}I%     \caption{Modelo de multirrotor proposto Bodie \cite{BODIEetal2020}}%     \label{fig:bodieconcept}% \end{figure}�Ainda no tarabalho de Bodie \cite{BODIEetal2020}, um protótipo foi construído e testado, e o trabalho apresenta os resultados.uEm Lu \textit{et al.} \cite{LUandYANGetal2020} a plataforma de estudo é um quadrirrotor com rotores direcionáveis. h\begin{figure}[H] %htbp (h) aqui, (t) topo, (b) base, ou na (p) página, ! - forçar, (H) - aqui "brabo"    \centering'        \subfigure[Referencial global.         \label{fig:lu1}]{E        \includegraphics[scale=0.45]{images/cap-2_RevisaoBib/Lu1.png}        } % end subfigure&        \subfigure[Referencial local.         \label{fig:lu2}]{D        \includegraphics[scale=0.4]{images/cap-2_RevisaoBib/Lu2.png}        } % end subfigure5        \quad % dá um espaço entre as duas figuras.S    \caption{Lu \textit{et al.} - Conceito e referenciais \cite{LUandYANGetal2020}}    \label{fig:luconceito}\end{figure}% \begin{figure}[H]%     \centering*%     \begin{subfigure}[b]{0.45\textwidth}%         \centeringG%         \includegraphics[scale=0.45]{images/cap-2_RevisaoBib/Lu1.png}'%         \caption{Referencial global.}%         \label{fig:lu1}%     \end{subfigure}%     \hfill*%     \begin{subfigure}[b]{0.45\textwidth}%         \centeringF%         \includegraphics[scale=0.4]{images/cap-2_RevisaoBib/Lu2.png}&%         \caption{Referencial local.}%         \label{fig:lu2}%     \end{subfigure}U%     \caption{Lu \textit{et al.} - Conceito e referenciais \cite{LUandYANGetal2020}}%     \label{fig:luconceito}% \end{figure}'Com o propósito de firmar as características de vôo omnidirecional, o principal objetivo do estudo de Lu \textit{et al.} \cite{LUandYANGetal2020} é desenvolver a arquitetura de controle e avaliar um observador de estados, com ênfase nas discussões sobre alocação de controle não-linear.h\begin{figure}[H] %htbp (h) aqui, (t) topo, (b) base, ou na (p) página, ! - forçar, (H) - aqui "brabo"    \centering%        \subfigure[Forças atuantes.         \label{fig:lu3}]{D        \includegraphics[scale=0.5]{images/cap-2_RevisaoBib/Lu3.png}        } % end subfigure'        \subfigure[Movimento simulado.         \label{fig:lu4}]{D        \includegraphics[scale=0.4]{images/cap-2_RevisaoBib/Lu4.png}        } % end subfigure5        \quad % dá um espaço entre as duas figuras.L    \caption{Lu \textit{et al.} - Forças e simulações \cite{CHENJIA2020}}    \label{fig:luforcas}\end{figure}%\vspace{1.5cm}% \begin{figure}[H]%     \centering*%     \begin{subfigure}[b]{0.45\textwidth}%         \centeringF%         \includegraphics[scale=0.5]{images/cap-2_RevisaoBib/Lu3.png}%%         \caption{Forças atuantes.}%         \label{fig:lu3}%     \end{subfigure}%     \hfill*%     \begin{subfigure}[b]{0.45\textwidth}%         \centeringF%         \includegraphics[scale=0.4]{images/cap-2_RevisaoBib/Lu4.png}'%         \caption{Movimento simulado.}%         \label{fig:lu4}%     \end{subfigure}N%     \caption{Lu \textit{et al.} - Forças e simulações \cite{CHENJIA2020}}%     \label{fig:luforcas}% \end{figure}�No trabalho de Chen e Jia \cite{CHENJIA2020}, apesar da utilização de rotores direcionáveis, foi sugerido a utlização de uma solução que mescla uma plataforma de asa fixa com o incremento dos rotores direcionáveis. 0\begin{figure}[H]%[!htb] = here or top or bottom    \centeringE    \includegraphics[scale=0.5]{images/cap-2_RevisaoBib/ChenJia1.png}P    \caption{Conceito estrutural desenvolvido por Chen e Jia \cite{CHENJIA2020}}    \label{fig:chenjia1}\end{figure}�Nesse trabalho, a plataforma é modelada sob o ponto de vista não linear e o problema de alocação de controle também é discutido. Os resultados contemplam comparações entre as simulações e o desempenho do protótipo em testes direcionados.2%\begin{figure}[!h]%[!htb] = here or top or bottom%    \centeringG%    \includegraphics[scale=0.75]{images/cap-2_RevisaoBib/ChenJia1.png}c%    \caption{Lu \textit{et al.} \cite{CHENJIA2020} - Forças atuantes - Fonte: \cite{CHENJIA2020}}%    \label{fig:luthrust}
%\end{figure}2%\begin{figure}[!h]%[!htb] = here or top or bottom%    \centeringH%    \includegraphics[scale=0.4]{images/cap-2_RevisaoBib/FranchiBi2.PNG}u%    \caption{Modelos conceituais avaliados por Franchi \textit{et al.} - Bicóptero - Fonte: \cite{FRANCHIetal2021}}%    \label{fig:franchi1}
%\end{figure}bApresentando uma plataforma com características semelhantes (asa-fixa combinada a atuação de asas rotativas direcionáveis) as desenvolvidas por Chen e Jia \cite{CHENJIA2020}, Gao \textit{et al.} \cite{GAOetal2020} desenvolve um estudo teórico sobre um método de detecção e diagnose de falhas empregando algoritmo de filtragem de Kalman Extendido.h\begin{figure}[H] %htbp (h) aqui, (t) topo, (b) base, ou na (p) página, ! - forçar, (H) - aqui "brabo"    \centeringE        \subfigure[Bicóptero. \label{fig:grafrestoquanton3d_t_cp1}]{T        \includegraphics[width=.4\textwidth]{images/cap-2_RevisaoBib/FranchiBi1.PNG}        } % end subfigure>        \subfigure[Tricóptero modelo 1. \label{fig:tricop1}]{U        \includegraphics[width=.4\textwidth]{images/cap-2_RevisaoBib/FranchiTri1.png}        } % end subfigure5        \quad % dá um espaço entre as duas figuras.>        \subfigure[Tricóptero modelo 2. \label{fig:tricop2}]{U        \includegraphics[width=.4\textwidth]{images/cap-2_RevisaoBib/FranchiTri2.PNG}        } % end subfigureB        \subfigure[Quadricóptero modelo 1. \label{fig:quadcop1}]{U        \includegraphics[width=.4\textwidth]{images/cap-2_RevisaoBib/FranchiQua1.PNG}        }% end subfigure5        \quad % dá um espaço entre as duas figuras.B        \subfigure[Quadricóptero modelo 2. \label{fig:quadcop2}]{V        \includegraphics[width=.4\textwidth]{images/cap-2_RevisaoBib/FranchiQua11.PNG}        } % end subfigureB        \subfigure[Quadricóptero modelo 3. \label{fig:quadcop3}]{V        \includegraphics[width=.4\textwidth]{images/cap-2_RevisaoBib/FranchiQua13.PNG}        } % end subfigure5        \quad % dá um espaço entre as duas figuras.B        \subfigure[Quadricóptero modelo 4. \label{fig:quadcop4}]{V        \includegraphics[width=.4\textwidth]{images/cap-2_RevisaoBib/FranchiQua15.PNG}        } % end subfigureB        \subfigure[Quadricóptero modelo 5. \label{fig:quadcop5}]{V        \includegraphics[width=.4\textwidth]{images/cap-2_RevisaoBib/FranchiQua14.PNG}"        }%%%%%%%%% % end subfigureo    \caption{Arquiteturas conceituais estudadas e avaliadas por Franchi \textit{et al.} \cite{FRANCHIetal2021}}    \label{fig:franchi2}\end{figure}% \begin{figure}[H]%     \centering)%     \begin{subfigure}[b]{0.4\textwidth}%         \centeringT%         \includegraphics[width=\textwidth]{images/cap-2_RevisaoBib/FranchiBi1.PNG}%         \caption{Bicóptero.}.%         \label{fig:grafrestoquanton3d_t_cp1}%     \end{subfigure}%     \quad)%     \begin{subfigure}[b]{0.4\textwidth}%         \centeringU%         \includegraphics[width=\textwidth]{images/cap-2_RevisaoBib/FranchiTri1.png})%         \caption{Tricóptero modelo 1.}%         \label{fig:tricop1}%     \end{subfigure}%     \quad)%     \begin{subfigure}[b]{0.4\textwidth}%         \centeringU%         \includegraphics[width=\textwidth]{images/cap-2_RevisaoBib/FranchiTri2.PNG})%         \caption{Tricóptero modelo 2.}%         \label{fig:tricop2}%     \end{subfigure}%     \quad)%     \begin{subfigure}[b]{0.4\textwidth}%         \centeringU%         \includegraphics[width=\textwidth]{images/cap-2_RevisaoBib/FranchiQua1.PNG},%         \caption{Quadricóptero modelo 1.}%         \label{fig:quadcop1}%     \end{subfigure}%     \quad)%     \begin{subfigure}[b]{0.4\textwidth}%         \centeringV%         \includegraphics[width=\textwidth]{images/cap-2_RevisaoBib/FranchiQua11.PNG},%         \caption{Quadricóptero modelo 2.}%         \label{fig:quadcop2}%     \end{subfigure}%     \quad)%     \begin{subfigure}[b]{0.4\textwidth}%         \centeringV%         \includegraphics[width=\textwidth]{images/cap-2_RevisaoBib/FranchiQua13.PNG},%         \caption{Quadricóptero modelo 3.}%         \label{fig:quadcop3}%     \end{subfigure}%     \quad)%     \begin{subfigure}[b]{0.4\textwidth}%         \centeringV%         \includegraphics[width=\textwidth]{images/cap-2_RevisaoBib/FranchiQua15.PNG},%         \caption{Quadricóptero modelo 4.}%         \label{fig:quadcop4}%     \end{subfigure}%     \quad)%     \begin{subfigure}[b]{0.4\textwidth}%         \centeringV%         \includegraphics[width=\textwidth]{images/cap-2_RevisaoBib/FranchiQua14.PNG},%         \caption{Quadricóptero modelo 5.}%         \label{fig:quadcop5}%     \end{subfigure}q%     \caption{Arquiteturas conceituais estudadas e avaliadas por Franchi \textit{et al.} \cite{FRANCHIetal2021}}%     \label{fig:franchi2}% \end{figure}
No trabalho de Franchi \textit{et al.} \cite{FRANCHIetal2021}, são analisados o impacto dos projetos de veículos aéreos multirotores em suas características e habilidades na execução de tarefas e manobras, com o enfoque nas propostas com potencial omnidirecional.2\begin{figure}[!ht]%[!htb] = here or top or bottom    \centeringH    \includegraphics[scale=0.5]{images/cap-2_RevisaoBib/FranchiTri3.PNG}�    \caption{Vista superior da representação cinemática conceitual de projetos coplanares com (da esquerda para a direita) hélices 4/6/8 \cite{FRANCHIetal2021}}    \label{fig:franchi3}\end{figure}PFranchi \cite{FRANCHIetal2021} propõe uma caracterização dos modelos de multirrotores encontrados na literatura e aplica um conjunto de avaliações e análises nessas topologias para veículos multirotores como pode ser visto nas Figs. \ref{fig:franchi2}, \ref{fig:franchi3}, \ref{fig:franchi4}, \ref{fig:franchi5} e \ref{fig:franchi6}. À partir da classificação de projetos da literatura, os mesmos são avaliados quanto as suas funcionalidades, ressaltando os relevantes detalhes construtivos, limitações e potencialidades, perpassando por uma grande gama de soluções possíveis.h\begin{figure}[H] %htbp (h) aqui, (t) topo, (b) base, ou na (p) página, ! - forçar, (H) - aqui "brabo"    \centeringB        \subfigure[Quadricóptero modelo 6. \label{fig:quadcop6}]{V        \includegraphics[width=.4\textwidth]{images/cap-2_RevisaoBib/FranchiQua16.PNG}        } % end subfigureB        \subfigure[Quadricóptero modelo 7. \label{fig:quadcop7}]{V        \includegraphics[width=.4\textwidth]{images/cap-2_RevisaoBib/FranchiQua17.PNG}        } % end subfigure5        \quad % dá um espaço entre as duas figuras.B        \subfigure[Quadricóptero modelo 8. \label{fig:quadcop8}]{V        \includegraphics[width=.4\textwidth]{images/cap-2_RevisaoBib/FranchiQua18.PNG}        } % end subfigureC        \subfigure[Pentacóptero modelo 1. \label{fig:pentacopt1}]{V        \includegraphics[width=.4\textwidth]{images/cap-2_RevisaoBib/FranchiQua19.PNG}        }% end subfigure�    \caption{Arquiteturas conceituais estudadas e avaliadas por Franchi \textit{et al.} \cite{FRANCHIetal2021} - Fonte: \cite{FRANCHIetal2021}}    \label{fig:franchi4}\end{figure}% \begin{figure}[H]%     \centering)%     \begin{subfigure}[b]{0.4\textwidth}%         \centeringV%         \includegraphics[width=\textwidth]{images/cap-2_RevisaoBib/FranchiQua16.PNG},%         \caption{Quadricóptero modelo 6.}%         \label{fig:quadcop6}%     \end{subfigure}%     \quad)%     \begin{subfigure}[b]{0.4\textwidth}%         \centeringV%         \includegraphics[width=\textwidth]{images/cap-2_RevisaoBib/FranchiQua17.PNG},%         \caption{Quadricóptero modelo 7.}%         \label{fig:quadcop7}%     \end{subfigure}%     \quad)%     \begin{subfigure}[b]{0.4\textwidth}%         \centeringV%         \includegraphics[width=\textwidth]{images/cap-2_RevisaoBib/FranchiQua18.PNG},%         \caption{Quadricóptero modelo 8.}%         \label{fig:quadcop8}%     \end{subfigure}%     \quad)%     \begin{subfigure}[b]{0.4\textwidth}%         \centeringV%         \includegraphics[width=\textwidth]{images/cap-2_RevisaoBib/FranchiQua19.PNG}+%         \caption{Pentacóptero modelo 1.} %         \label{fig:pentacopt1}%     \end{subfigure}�%     \caption{Arquiteturas conceituais estudadas e avaliadas por Franchi \textit{et al.} \cite{FRANCHIetal2021} - Fonte: \cite{FRANCHIetal2021}}%     \label{fig:franchi4}% \end{figure}�Além disso, Franchi \cite{FRANCHIetal2021} discute as principais características que comumente se destacam em cada projeto, como por exemplo, o foco em simetria, o uso de hélices unidirecionais, ignorando interação aerodinâmica entre hélices, e por fim a modelagem do sistema, desconsiderando os limites de atuação, destacando inclusive que as avaliações ainda encontram-se em voga, dado a necessidade de aprofundamento nessa temática. h\begin{figure}[H] %htbp (h) aqui, (t) topo, (b) base, ou na (p) página, ! - forçar, (H) - aqui "brabo"    \centeringA        \subfigure[Hexacóptero modelo 1. \label{fig:hexacopt1}]{V        \includegraphics[width=.4\textwidth]{images/cap-2_RevisaoBib/FranchiQua23.PNG}        } % end subfigureA        \subfigure[Hexacóptero modelo 2. \label{fig:hexacopt2}]{V        \includegraphics[width=.4\textwidth]{images/cap-2_RevisaoBib/FranchiQua25.PNG}        } % end subfigure5        \quad % dá um espaço entre as duas figuras.A        \subfigure[Hexacóptero modelo 3. \label{fig:hexacopt3}]{V        \includegraphics[width=.4\textwidth]{images/cap-2_RevisaoBib/FranchiQua26.PNG}        } % end subfigureC        \subfigure[Heptacóptero modelo 1. \label{fig:heptacopt1}]{V        \includegraphics[width=.4\textwidth]{images/cap-2_RevisaoBib/FranchiQua27.PNG}"        }%%%%%%%%% % end subfigureo    \caption{Arquiteturas conceituais estudadas e avaliadas por Franchi \textit{et al.} \cite{FRANCHIetal2021}}    \label{fig:franchi5}\end{figure}% \begin{figure}[H]%     \centering)%     \begin{subfigure}[b]{0.4\textwidth}%         \centeringV%         \includegraphics[width=\textwidth]{images/cap-2_RevisaoBib/FranchiQua23.PNG}*%         \caption{Hexacóptero modelo 1.}%         \label{fig:hexacopt1}%     \end{subfigure}%     \quad)%     \begin{subfigure}[b]{0.4\textwidth}%         \centeringV%         \includegraphics[width=\textwidth]{images/cap-2_RevisaoBib/FranchiQua25.PNG}*%         \caption{Hexacóptero modelo 2.}%         \label{fig:hexacopt2}%     \end{subfigure}%     \quad)%     \begin{subfigure}[b]{0.4\textwidth}%         \centeringV%         \includegraphics[width=\textwidth]{images/cap-2_RevisaoBib/FranchiQua26.PNG}*%         \caption{Hexacóptero modelo 3.}%         \label{fig:hexacopt3}%     \end{subfigure}%     \quad)%     \begin{subfigure}[b]{0.4\textwidth}%         \centeringV%         \includegraphics[width=\textwidth]{images/cap-2_RevisaoBib/FranchiQua27.PNG}+%         \caption{Heptacóptero modelo 1.} %         \label{fig:heptacopt1}%     \end{subfigure}q%     \caption{Arquiteturas conceituais estudadas e avaliadas por Franchi \textit{et al.} \cite{FRANCHIetal2021}}%     \label{fig:franchi5}% \end{figure}�Em particular, Franchi \cite{FRANCHIetal2021} aborda o impacto do problema de simetria da arquitetura do sistema e os aborda ressaltando os aspectos de otimização da geração de empuxo à partir da disposição das hélices, discutindo inclusive a não consideração desta, promovendo os resultados da literatura e destacando que esse caminho envolvendo a otimização das posições dos motores tem potencial em extender a fronteira do tema. Também discute a possibilidade de emprego de hélices bidirecionais, a interação aerodinâmica entre as hélices e os limites dos atuadores, inclusive fazendo menção aos trabalhos de D'Andreas e Brescianini \cite{DEANDREASBRESCIANINI2016}\cite{DANDREASBRESCIANINI2018}.h\begin{figure}[H] %htbp (h) aqui, (t) topo, (b) base, ou na (p) página, ! - forçar, (H) - aqui "brabo"    \centering=        \subfigure[Octacóptero modelo 1. \label{fig:octa1}]{V        \includegraphics[width=.4\textwidth]{images/cap-2_RevisaoBib/FranchiOct31.PNG}        } % end subfigure=        \subfigure[Octacóptero modelo 2. \label{fig:octa2}]{V        \includegraphics[width=.4\textwidth]{images/cap-2_RevisaoBib/FranchiOct32.PNG}"        }%%%%%%%%% % end subfigureo    \caption{Arquiteturas conceituais estudadas e avaliadas por Franchi \textit{et al.} \cite{FRANCHIetal2021}}    \label{fig:franchi6}\end{figure}% \begin{figure}[H]%     \centering)%     \begin{subfigure}[b]{0.4\textwidth}%         \centeringV%         \includegraphics[width=\textwidth]{images/cap-2_RevisaoBib/FranchiOct31.PNG}*%         \caption{Octacóptero modelo 1.}%         \label{fig:octa1}%     \end{subfigure}%     \quad)%     \begin{subfigure}[b]{0.4\textwidth}%         \centeringV%         \includegraphics[width=\textwidth]{images/cap-2_RevisaoBib/FranchiOct32.PNG}*%         \caption{Octacóptero modelo 2.}%         \label{fig:octa2}%     \end{subfigure}q%     \caption{Arquiteturas conceituais estudadas e avaliadas por Franchi \textit{et al.} \cite{FRANCHIetal2021}}%     \label{fig:franchi6}% \end{figure}Já nas propostas de projeto que envolvem o desenvolvimento dos multirrotores omnidirecionais com os rotores fixos, alguns trabalhos que se dedicam a esse tema e seus correlatos são os trabalhos de D'Andreas e Brescianini \cite{DEANDREASBRESCIANINI2016}\cite{DANDREASBRESCIANINI2018}, de Kovac e Hamaza \cite{KOVACHAMAZA2018}, de Dyer \cite{PhDThesisDYER2018}, de Caccavele \textit{et al.} \cite{CACCAVALEetal2019}, e também os trabalhos de Nokleby \textit{et al.} \cite{NOKLEBYetal2019} e Baird e Nokleby \cite{BAIRDNOKLEBY2020}.�Os trabalhos de D'Andreas e Brescianini  \cite{DEANDREASBRESCIANINI2016}\cite{DANDREASBRESCIANINI2018} cerram a sua atenção para um VANT Multirrotor de configuração tridimensional, sobreatuado, com rotores fixos em posições ótimas, cujas posições estam inscritas nas diagonais de um cubo unitário, de forma a entregar o máximo empuxo disponível em todas as direções, à partir dos seus oito motores (Fig. \ref{fig:brescianiniconcept1}). h\begin{figure}[H] %htbp (h) aqui, (t) topo, (b) base, ou na (p) página, ! - forçar, (H) - aqui "brabo"    \centering;        \subfigure[Octacóptero. \label{fig:brescianini1}]{X        \includegraphics[width=.55\textwidth]{images/cap-2_RevisaoBib/Brescianini02.PNG}        } % end subfigureJ        \subfigure[Octacóptero - referenciais. \label{fig:brescianini2}]{X        \includegraphics[width=.325\textwidth]{images/cap-2_RevisaoBib/Brescianini4.PNG}"        }%%%%%%%%% % end subfigure}    \caption{Proposta desenvolvida por D'Andreas e Brescianini \cite{DEANDREASBRESCIANINI2016}\cite{DANDREASBRESCIANINI2018}}#    \label{fig:brescianiniconcept1}\end{figure}% \begin{figure}[H]%     \centering*%     \begin{subfigure}[b]{0.55\textwidth}%         \centeringW%         \includegraphics[width=\textwidth]{images/cap-2_RevisaoBib/Brescianini02.PNG}!%         \caption{Octacóptero.}"%         \label{fig:brescianini1}%     \end{subfigure}%     \quad+%     \begin{subfigure}[b]{0.325\textwidth}%         \centeringV%         \includegraphics[width=\textwidth]{images/cap-2_RevisaoBib/Brescianini4.PNG}0%         \caption{Octacóptero - referenciais.}"%         \label{fig:brescianini2}%     \end{subfigure}%     \caption{Proposta desenvolvida por D'Andreas e Brescianini \cite{DEANDREASBRESCIANINI2016}\cite{DANDREASBRESCIANINI2018}}%     \label{fig:brescianini1}% \end{figure};Ainda segundo D'Andreas e Brescianini \cite{DEANDREASBRESCIANINI2016}\cite{DANDREASBRESCIANINI2018}, essa solução obtida à partir de processos de otimização se mostrou promissora em desacoplar a dinâmica, o que permitiu que o protótipo performasse de forma satisfatória e conforme previsto nas simulações.Z% TALVEZ SEJA O CASO DE INCLUIR OS DETALHES DO PROCESSO DE OTIMIZAÇÃO SUGERIDO POR ELES.2\begin{figure}[!ht]%[!htb] = here or top or bottom    \centeringK    \includegraphics[scale=0.45]{images/cap-2_RevisaoBib/KovacHamaza01.PNG}n    \caption{Omni-drone - Conceito proposto e desenvolvido por Kovac e Hamaza - Fonte: \cite{KOVACHAMAZA2018}}    \label{fig:omnidrone1}\end{figure}�Em Kovac e Hamaza \cite{KOVACHAMAZA2018} apresenta a integração de uma plataforma aérea comum com quatro motores e um manipulador composto por um mecanismo de cinco barras, de forma que a interação com o espaço de trabalho seja omini-direcional, isto é, o leiaute oferece um espaço de trabalho omnidirecional, aumentando a versatilidade do sistema aéreo e as tarefas possíveis. �Nesse trabalho é apresentado o conceito à partir do projeto mecânico, da análise da cinemática e o estudo do espaço de trabalho, com ênfase nos casos envolvendo interação aérea com tetos, superfícies curvas e interação lateral com fachadas.2\begin{figure}[!ht]%[!htb] = here or top or bottom    \centeringJ    \includegraphics[scale=0.4]{images/cap-2_RevisaoBib/KovacHamaza02.PNG}e    \caption{Omni-drone - Conceito proposto e desenvolvido por Kovac e Hamaza \cite{KOVACHAMAZA2018}}    \label{fig:omnidrone2}\end{figure}�Dyer \cite{PhDThesisDYER2018} em seu trabalho apresenta uma plataforma muito semelhante a desenvolvida por D'Andreas e Brescianini \cite{DEANDREASBRESCIANINI2016}\cite{DANDREASBRESCIANINI2018}. Nesse trabalho o desenvolvimento se deu essencialmente nos tópicos envolvendo as configurações ótimas para as hélices, na modelagem do comportamento das hélices e na alocação de controle, com a discussão dos resultados obtidos.2\begin{figure}[!ht]%[!htb] = here or top or bottom    \centeringL    \includegraphics[width=.4\textwidth]{images/cap-2_RevisaoBib/Dyer02.PNG}g    \caption{Omni-directional UAV - Conceito proposto e desenvolvido por Dyer \cite{PhDThesisDYER2018}}    \label{fig:dyer1}\end{figure}h\begin{figure}[H] %htbp (h) aqui, (t) topo, (b) base, ou na (p) página, ! - forçar, (H) - aqui "brabo"    \centering@        \subfigure[Omni-directional UAV. \label{fig:omnidyer1}]{P        \includegraphics[width=.6\textwidth]{images/cap-2_RevisaoBib/Dyer01.PNG}        } % end subfigure@        \subfigure[Omni-directional UAV. \label{fig:omnidyer2}]{Q        \includegraphics[width=.35\textwidth]{images/cap-2_RevisaoBib/Dyer03.PNG}"        }%%%%%%%%% % end subfigure
        \quadC        \subfigure[Visão geral do sistema. \label{fig:omnidyer3}]{Q        \includegraphics[width=.95\textwidth]{images/cap-2_RevisaoBib/Dyer04.PNG}"        }%%%%%%%%% % end subfigureg    \caption{Omni-directional UAV - Conceito proposto e desenvolvido por Dyer \cite{PhDThesisDYER2018}}    \label{fig:dyer2}\end{figure}% \begin{figure}[H]%     \centering)%     \begin{subfigure}[b]{0.6\textwidth}%         \centeringP%         \includegraphics[width=\textwidth]{images/cap-2_RevisaoBib/Dyer01.PNG})%         \caption{Omni-directional UAV.}%         \label{fig:omnidyer1}%     \end{subfigure}%     \quad*%     \begin{subfigure}[b]{0.35\textwidth}%         \centeringP%         \includegraphics[width=\textwidth]{images/cap-2_RevisaoBib/Dyer03.PNG})%         \caption{Omni-directional UAV.}%         \label{fig:omnidyer2}%     \end{subfigure}%     \quad*%     \begin{subfigure}[b]{0.95\textwidth}%         \centeringP%         \includegraphics[width=\textwidth]{images/cap-2_RevisaoBib/Dyer04.PNG},%         \caption{Visão geral do sistema.}%         \label{fig:omnidyer3}%     \end{subfigure}i%     \caption{Omni-directional UAV - Conceito proposto e desenvolvido por Dyer \cite{PhDThesisDYER2018}}%     \label{fig:dyer2}% \end{figure}(Já no conceito proposto por Nigro \textit{et al.} \cite{NIGROetal2019}, apesar de não se lançar como uma solução omnidirecional, é apresentada como uma plataforma plenamente atuada, apesar de ser idealizada com apenas quatro rotores (Figs. \ref{fig:nigroconcept1} e \ref{fig:nigrogeral2}). h\begin{figure}[H] %htbp (h) aqui, (t) topo, (b) base, ou na (p) página, ! - forçar, (H) - aqui "brabo"    \centering;        \subfigure[Detalhe do sistema. \label{fig:nigro1}]{Q        \includegraphics[width=.4\textwidth]{images/cap-2_RevisaoBib/Nigro01.PNG}        } % end subfigure;        \subfigure[Detalhe do sistema. \label{fig:nigro2}]{R        \includegraphics[width=.55\textwidth]{images/cap-2_RevisaoBib/Nigro03.PNG}"        }%%%%%%%%% % end subfigure>    \caption{Conceito proposto por Nigro \cite{NIGROetal2019}}    \label{fig:nigroconcept1}\end{figure}% \begin{figure}[H]%     \centering)%     \begin{subfigure}[b]{0.4\textwidth}%         \centeringQ%         \includegraphics[width=\textwidth]{images/cap-2_RevisaoBib/Nigro01.PNG}'%         \caption{Detalhe do sistema.}%         \label{fig:nigro1}%     \end{subfigure}%     \quad*%     \begin{subfigure}[b]{0.55\textwidth}%         \centeringQ%         \includegraphics[width=\textwidth]{images/cap-2_RevisaoBib/Nigro03.PNG}'%         \caption{Detalhe do sistema.}%         \label{fig:nigro2}%     \end{subfigure}@%     \caption{Conceito proposto por Nigro \cite{NIGROetal2019}}%     \label{fig:nigro1}% \end{figure}2\begin{figure}[!ht]%[!htb] = here or top or bottom    \centeringE    \includegraphics[scale=0.65]{images/cap-2_RevisaoBib/Nigro04.PNG}:    \caption{Visão geral do sistema \cite{NIGROetal2019}}    \label{fig:nigrogeral2}\end{figure}_Nesse conceito os rotores permanecem fixos no chassi e esse se orienta em relação ao seu núcleo/\textit{payload} a depender da atitude desejada. Algumas outras características são ressaltadas, como a capacidade de minimizar os esforços internos e a dissipação de energia devido à algumas condições mecânicas (Fig. \ref{fig:nigrodetalhe2}) 2\begin{figure}[!ht]%[!htb] = here or top or bottom    \centeringE    \includegraphics[scale=0.65]{images/cap-2_RevisaoBib/Nigro02.PNG}?    \caption{Carga com atitude invariante \cite{NIGROetal2019}}    \label{fig:nigrodetalhe2}\end{figure}�Nos trabalhos de Nokleby \textit{et al.} \cite{NOKLEBYetal2019} e Baird e Nokleby \cite{BAIRDNOKLEBY2020} são apresentados o desenvolvimento, prototipagem e testes do omnicopter e do omniraptor.h\begin{figure}[H] %htbp (h) aqui, (t) topo, (b) base, ou na (p) página, ! - forçar, (H) - aqui "brabo"    \centeringC        \subfigure[Visão geral do sistema. \label{fig:nokleby1a}]{S        \includegraphics[width=.35\textwidth]{images/cap-2_RevisaoBib/Nokleby1.png}        } % end subfigure6        \subfigure[Omniraptor. \label{fig:nokleby1b}]{R        \includegraphics[width=.6\textwidth]{images/cap-2_RevisaoBib/Nokleby2.png}"        }%%%%%%%%% % end subfigure{    \caption{Visão geral do sistema proposto por Nokleby \cite{NOKLEBYetal2019} e Baird e Nokleby \cite{BAIRDNOKLEBY2020}}    \label{fig:nokleby1}\end{figure}% \begin{figure}[H]%     \centering*%     \begin{subfigure}[b]{0.35\textwidth}%         \centeringR%         \includegraphics[width=\textwidth]{images/cap-2_RevisaoBib/Nokleby1.png},%         \caption{Visão geral do sistema.}%         \label{fig:nokleby1a}%     \end{subfigure}%     \quad)%     \begin{subfigure}[b]{0.6\textwidth}%         \centeringR%         \includegraphics[width=\textwidth]{images/cap-2_RevisaoBib/Nokleby2.png}%         \caption{Omniraptor.}%         \label{fig:nokleby1b}%     \end{subfigure}}%     \caption{Visão geral do sistema proposto por Nokleby \cite{NOKLEBYetal2019} e Baird e Nokleby \cite{BAIRDNOKLEBY2020}}%     \label{fig:nokleby1}% \end{figure}Enquanto que em Nokleby \textit{et al.} \cite{NOKLEBYetal2019} discute-se o conceito, o projeto, o protótipo e suas pontencialidades à partir de simulações e testes de performance, em Baird e Nokleby \cite{BAIRDNOKLEBY2020} é focado no desenvolvimento no dispositivo de ancoragem. h\begin{figure}[H] %htbp (h) aqui, (t) topo, (b) base, ou na (p) página, ! - forçar, (H) - aqui "brabo"    \centeringC        \subfigure[Visão geral do sistema. \label{fig:nokleby2a}]{R        \includegraphics[width=.5\textwidth]{images/cap-2_RevisaoBib/Nokleby6.png}        } % end subfigure6        \subfigure[Omniraptor. \label{fig:nokleby2b}]{T        \includegraphics[width=.425\textwidth]{images/cap-2_RevisaoBib/Nokleby3.png}"        }%%%%%%%%% % end subfigure{    \caption{Visão geral do sistema proposto por Nokleby \cite{NOKLEBYetal2019} e Baird e Nokleby \cite{BAIRDNOKLEBY2020}}    \label{fig:nokleby2}\end{figure}% \begin{figure}[H]%     \centering)%     \begin{subfigure}[b]{0.5\textwidth}%         \centeringR%         \includegraphics[width=\textwidth]{images/cap-2_RevisaoBib/Nokleby6.png},%         \caption{Visão geral do sistema.}%         \label{fig:nokleby2a}%     \end{subfigure}%     \quad+%     \begin{subfigure}[b]{0.425\textwidth}%         \centeringR%         \includegraphics[width=\textwidth]{images/cap-2_RevisaoBib/Nokleby3.png}%         \caption{Omniraptor.}%         \label{fig:nokleby2b}%     \end{subfigure}}%     \caption{Visão geral do sistema proposto por Nokleby \cite{NOKLEBYetal2019} e Baird e Nokleby \cite{BAIRDNOKLEBY2020}}%     \label{fig:nokleby2}% \end{figure}uAinda nesse segundo trabalho, testes de operação são conduzidos mostrando a capacidade e robustez da metodologia. h\begin{figure}[H] %htbp (h) aqui, (t) topo, (b) base, ou na (p) página, ! - forçar, (H) - aqui "brabo"    \centering;        \subfigure[Teste realizado. \label{fig:nokleby2c}]{S        \includegraphics[width=.45\textwidth]{images/cap-2_RevisaoBib/Nokleby7.png}        } % end subfigure@        \subfigure[Sistema de ancoragem. \label{fig:nokleby2d}]{R        \includegraphics[width=.5\textwidth]{images/cap-2_RevisaoBib/Nokleby5.png}"        }%%%%%%%%% % end subfigure
        \quad7        \subfigure[Performance. \label{fig:nokleby2e}]{R        \includegraphics[width=.5\textwidth]{images/cap-2_RevisaoBib/Nokleby4.png}	        }7        \subfigure[Performance. \label{fig:nokleby2f}]{S        \includegraphics[width=.45\textwidth]{images/cap-2_RevisaoBib/Nokleby8.png}	        }�    \caption{Teste de operação do sistema de ancoragem e performance do sistema proposto por Nokleby \cite{NOKLEBYetal2019} e Baird e Nokleby \cite{BAIRDNOKLEBY2020}}    \label{fig:nokleby3}\end{figure}% \begin{figure}[H]%     \centering*%     \begin{subfigure}[b]{0.45\textwidth}%         \centeringR%         \includegraphics[width=\textwidth]{images/cap-2_RevisaoBib/Nokleby7.png}$%         \caption{Teste realizado.}%         \label{fig:nokleby2a}%     \end{subfigure}%     \quad)%     \begin{subfigure}[b]{0.5\textwidth}%         \centeringR%         \includegraphics[width=\textwidth]{images/cap-2_RevisaoBib/Nokleby5.png})%         \caption{Sistema de ancoragem.}%         \label{fig:nokleby2b}%     \end{subfigure}%     \quad)%     \begin{subfigure}[b]{0.5\textwidth}%         \centeringR%         \includegraphics[width=\textwidth]{images/cap-2_RevisaoBib/Nokleby4.png} %         \caption{Performance.}%         \label{fig:nokleby2c}%     \end{subfigure}%     \quad*%     \begin{subfigure}[b]{0.45\textwidth}%         \centeringR%         \includegraphics[width=\textwidth]{images/cap-2_RevisaoBib/Nokleby8.png} %         \caption{Performance.}%         \label{fig:nokleby2d}%     \end{subfigure}�%     \caption{Teste de operação do sistema de ancoragem e performance do sistema proposto por Nokleby \cite{NOKLEBYetal2019} e Baird e Nokleby \cite{BAIRDNOKLEBY2020}}%     \label{fig:nokleby3}% \end{figure}gÉ possível resumir todas as abordagens desses trabalhos em ao menos cinco grandes tópicos, à saber:0a) Modelagem Matemática - Dinâmica e Controle;b) Dinâmica desacoplada;/c) Estratégias de Controle aplicadas em VANTs;d) Alocação de controle; ee) Formas de atuação.�Além desses, todos os arranjos verificados nos trabalhos podem ser agrupados em ao menos dois grandes grupos de arranjo físicos:@a) VANTs com motores fixos (rotores simples ou rotores duplo); eHb) VANTS com motores direcionáveis (rotores simples ou rotores duplos).�%http://latexbr.blogspot.com/2011/07/inserindo-figuras-no-latex.html#:~:text=Inserindo%20as%20figuras&text=%5B!,ficar%C3%A1%20exatamente%20onde%20voc%C3%AA%20inseriu. %\subsection{Estado da Técnica} % (ver Fig. \ref{fig:my_label}).+%Segundo Ye \emph{et al.} \cite{YEetal2013}%\begin{figure} %   \centering. %   \includegraphics{images/imagem_teste.png} %   \caption{Caption} %   \label{fig:my_label}
%\end{figure}%\begin{figure} %   \centering9 %   \includegraphics[scale=0.5]{images/imagem_teste.png}! %   \caption{Essa figure é ...} %   \label{fig:outro_label}
%\end{figure}$%section{Referência Bibliográfica}5��